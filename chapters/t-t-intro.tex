\chapter*{Введение}
\label{sec:afterwords}
\addcontentsline{toc}{chapter}{Введение}

В настоящее время актуальны проблемы анализа многомерных данных, особенно в медицинских приложениях.
Данная работа рассматривает новый подход задаче понижения размерности: матрицу Межиндивидуальных корреляций. Экспертами в этой области автор считает Juha Pajula из университета Тампере в 2016 году защитившего диссертацию по данной теме \cite{pajula_ISC_thes}, Jukka-Pekka Kauppi \cite{kauppi2010clustering}~--- первое упоминание применения метода для задачи кластеризации данных фМРТ.
% Первая публикаци по теме увидела свет в 2010 году \cite{}.
Однако в этой работе рассматривались естественные стимулы (просмотр фильмов) что не соотносится с доступными авторам данными (вербальные и пространственные задачи). Из-за разреженности данных без модификации методы
предыдущих исследований не применимы без модификаций.

Таким образом, получим задачу данной работы~--- использование метода межиндивидуальной корреляции для кластеризации данных в задаче понижения размерности. Также проведено сравнение нового метода с традиционными подходами к данной задаче, не использующими множество испытуемых (Метод главных компонент (PCA), \ldots).

В первой главе подробно рассматриваются теоретические аспекты задачи понижения размерности, задачи классификации (Метод опорных векторов (SVM), нейронные сети, Линейный дискриминантный анализ (ЛДА)) и специфических для проблемной области (фМРТ) подходов к анализу данных. Также описываются программные средства визуализации трёхмерных данных с примерами их использования (nilearn.plotting,~matplotlib3d,~NIFTI)


Введение всегда содержит краткую характеристику работы по следующим аспектам:

\begin{itemize}
	\item актуальность:
	\begin{itemize}
		\item кто и почему в настоящее время интересуется данной проблематикой (в т.ч. для решения каких задач могут быть полезны исслелования в данной области),
		\item краткая история вопроса (в формате год-фамилия-что сделал),
		\item нерешенные вопросы/проблемы;
	\end{itemize}
	\item новизна работы (что нового привносится данной работой);
	\item оригинальная суть исследования;
	\item содержание по главам (по одному абзацу на главу).
\end{itemize}

Общий объем введения должен не превышать 1,5 страниц (для ПЗ к УИРам может быть чуть меньше).


