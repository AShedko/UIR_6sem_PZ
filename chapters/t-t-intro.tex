\chapter*{Введение}
\label{sec:afterwords}
\addcontentsline{toc}{chapter}{Введение}

Введение всегда содержит краткую характеристику работы по следующим аспектам:

\begin{itemize}
	\item актуальность:
	\begin{itemize}
		\item кто и почему в настоящее время интересуется данной проблематикой (в т.ч. для решения каких задач могут быть полезны исслелования в данной области),
		\item краткая история вопроса (в формате год-фамилия-что сделал),
		\item нерешенные вопросы/проблемы;
	\end{itemize}
	\item новизна работы (что нового привносится данной работой);
	\item оригинальная суть исследования;
	\item содержание по главам (по одному абзацу на главу).
\end{itemize}

Общий объем введения должен не превышать 1,5 страниц (для ПЗ к УИРам может быть чуть меньше).\cite{pajula_ISC_tut}


