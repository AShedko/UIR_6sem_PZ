\chapter*{Введение}
\label{sec:afterwords}
\addcontentsline{toc}{chapter}{Введение}

В настоящее время актуальны проблемы анализа многомерных данных, особенно в медицинских приложениях.
Данная работа рассматривает новый подход задаче понижения размерности: матрицу Межиндивидуальных корреляций. Акттивно публикуются в этой области: Juha Pajula из университета Тампере~(в 2016 году защитившего диссертацию по данной теме \cite{pajula_ISC_thes}\,), Jussi Tohka, Jukka-Pekka Kauppi, Юрия Хассона, впервые описавшего данный метод. Первое упоминание применения метода для задачи кластеризации данных фМРТ можно найти в статье Юрия Хассона и других в 2004\cite{Hasson1634}.
Однако в этой работе рассматривались естественные стимулы (просмотр фильмов) что не соотносится с доступными авторам данными (вербальные и пространственные задачи). Из-за разреженности данных без модификации методы
предыдущих исследований не применимы без модификаций.

Таким образом, получим задачу данной работы~--- использование метода межиндивидуальной корреляции для кластеризации данных в задаче понижения размерности. Также проводится сравнение нового метода с традиционными подходами к данной задаче, не использующими множество испытуемых ( T-статистика, Обобщённая линейная модель (GLM)).

Новизна работы состоит в применении метода ISC для кластеризации в условиях неестественных стимулов.

В первой главе подробно рассматриваются теоретические аспекты задачи понижения размерности, задачи классификации (Метод опорных векторов (SVM), нейронные сети, Линейный дискриминантный анализ (ЛДА)) и специфических для проблемной области (фМРТ) подходов к анализу данных. Также описываются программные средства визуализации трёхмерных данных с примерами их использования. (\verb|nilearn.plotting|\cite{10.3389/fninf.2014.00014}, \verb|matplotlib3d|\cite{Hunter:2007}, \verb|NIFTI|, \verb|MITK|\cite{wolf2004medical})

Во второй главе описаны используемые в работе алгоритмы, а именно: кластеризация на основе ISC, формирование вектора признаков, вычисление показателей точности классификации, классификация в режиме реального времени.

В третьей главе рассматриваются программные аспекты реализации алгоритмов описанных в предыдущей главе.

В заключительной главе описывается характер экспериментальных данных и количественные показатели точности работы системы. Также проводится исследование эффективности различных показателей точности классификации применительно к конкретным экспериментальным данным.

%Введение всегда содержит краткую характеристику работы по следующим аспектам:
%
%\begin{itemize}
%	\item актуальность:
%	\begin{itemize}
%		\item кто и почему в настоящее время интересуется данной проблематикой (в т.ч. для решения каких задач могут быть полезны исслелования в данной области),
%		\item краткая история вопроса (в формате год-фамилия-что сделал),
%		\item нерешенные вопросы/проблемы;
%	\end{itemize}
%	\item новизна работы (что нового привносится данной работой);
%	\item оригинальная суть исследования;
%	\item содержание по главам (по одному абзацу на главу).
%\end{itemize}
%
%Общий объем введения должен не превышать 1,5 страниц (для ПЗ к УИРам может быть чуть меньше).


