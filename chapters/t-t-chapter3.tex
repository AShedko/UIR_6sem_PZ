\chapter{Разработка программной системы для классификации сигналов фМРТ}


\section{Проектирование программного пакета выполняющего классификацию когнитивных состояний по данным фМРТ на основе анализа межиндивидуальных корреляций}


\section{Программная реализация системы классификации}
\begin{annotation}
	Здесь будет описана реализация
	\begin{itemize}
		\item алгоритмов быстрой (параллельной загрузки) примеров для обучения и кластеризации.		
	\end{itemize}

\end{annotation}
В этом разделе обосновывается выбор инструментальных средств; одним из критериев выбора могут быть какие-либо требования к разрабатываемой системе, и если этих требований много, они могут быть выделены в отдельный раздел, или же в приложение. Этот пункт не пишется, если в аналитической главе был раздел, посвященный сравнительному анализу и выбору инструментальных средств.




\section{Состав и структура реализованного программного обеспечения}
\begin{annotation}
	Разработанное приложение является подключаемой библиотекой для использования в среде "интерактивных тетрадей" \texttt{jupyter}. В состав библиотеки входят:
	\begin{itemize}
		\item Модуль параллельной загрузки/выгрузки примеров/изображений
		\item Модуль кластеризаци, содержащий алогритмы, описанные в части 2.
		\item Модуль классификации, предоставляющий на выбор несколько классификаторов и их входные параметры.
		\item Модуль визуализации для удобного представления данных для последующегог анализа специалистом		
	\end{itemize}
\end{annotation}

\section{Основные сценарии работы пользователя}
\begin{annotation}
	Подразумевается следующий сценарий работы: пользователь подключатся к серверу интерактивных рабочих тетрадей декларативно описывает свои действия. Сценарий подразумевает загрузку дополнительных обучающих/тестовых выборок для проверки корректности работы алгоритмов.
\end{annotation}

\section{Сравнение реализованного программного обеспечения с существующими аналогами}
\begin{annotation}
	Автор не нашёл аналогов данного приложения по причине низкой востребованности.
\end{annotation}

В сравнении должно быть отражено, чем полученное ПО выгодно (и невыгодно) отличается от прочих ближайших аналогов. Практика показывает, что аналоги есть всегда. А если нет аналогов, значит есть частичные решения, которые реализуют какие-то части функционала вашей системы. Тут тоже может быть относительно много таблиц и графиков.



\section{Выводы}

Следует перечислить, какие практические результаты были получены, а именно: какое программное или иное обеспечение было создано. В число результатов могут входить, например, методики тестирования, тестовые примеры (для проверки корректности/оценки характеристик тех или иных алгоритмов) и др. По каждому результату следует сделать вывод, насколько он отличается от известных промышленных аналогов и исследовательских прототипов.

