\chapter{Анализ проблематики задач классификации когнитивных состояний}
\label{chapter1}
\begin{annotation}
	В первой главе подробно рассматриваются теоретические аспекты задачи понижения размерности, задачи классификации (Метод опорных векторов (SVM), нейронные сети, Линейный дискриминантный анализ (ЛДА)) и специфических для проблемной области (фМРТ) подходов к анализу данных. Также описываются программные средства визуализации трёхмерных данных с примерами их использования. (\verb|nilearn.plotting|\cite{10.3389/fninf.2014.00014}, \verb|matplotlib3d|\cite{Hunter:2007}, \verb|NIFTI|, \verb|MITK|\cite{wolf2004medical})
\end{annotation}

\section{Изучение и анализ подходов к классификации когнитивных состояний по данным фМРТ (статическим и динамическим) применительно к задачам медицинской диагностики}



\section{Сравнительный анализ методов классификации многомерных данных }
\begin{annotation}
	Рассмотрим такие методы как: Метод опорных векторов (SVM), нейронные сети, Линейный дискриминантный анализ (ЛДА)
\end{annotation}

Вначале дадим общее определение задачи классификации.
Пусть входные данные имеют вид $\left\{(x_i,c_i)| i \in \overline{1,P}\right\}$, где $P$--- количество примеров, $x_i \in \mathcal X, c_i \in \mathcal C$, $\mathcal X$-- пространство входных признаков, $\mathcal C$-- конечное множество классов. Часто множество $\mathcal C$ является 2 элементным, в этом случае классификация называется \textit{бинарной}. Требуется построит функцию, которая для каждого $x \in \mathcal X$  даёт правильное значение $c$. 
\subsection{SVM}

Рассмотрим метод опорных векторов, для чего вначале дадим определение метода.

\textbf{Метод опорных векторов} (\emph {англ. SVM, support vector machine}) — набор схожих алгоритмов обучения с учителем, использующихся для задач классификации и регрессионного анализа. Основная идея метода — перевод исходных векторов в пространство более высокой размерности и поиск разделяющей гиперплоскости с максимальным зазором в этом пространстве. Две параллельных гиперплоскости строятся по обеим сторонам гиперплоскости, разделяющей классы. Разделяющей гиперплоскостью будет гиперплоскость, максимизирующая расстояние до двух параллельных гиперплоскостей. Алгоритм работает в предположении, что чем больше разница или расстояние между этими параллельными гиперплоскостями, тем меньше будет средняя ошибка классификатора.




\section{Сравнительный анализ программных средств визуализации трехмерных данных фМРТ и исследование возможности их использования}
\begin{annotation}
	В разделе описаны  различные программные компоненты для визуализации нейро-данных.
\end{annotation}

\subsection{Nilearn}
Nilearn makes it easy to use many advanced machine learning, pattern recognition and multivariate statistical techniques on neuroimaging data for applications such as MVPA (Mutli-Voxel Pattern Analysis), decoding, predictive modelling, functional connectivity, brain parcellations, connectomes.

Nilearn can readily be used on task fMRI, resting-state, or VBM data.

For a machine-learning expert, the value of nilearn can be seen as domain-specific feature engineering construction, that is, shaping neuroimaging data into a feature matrix well suited to statistical learning, or vice versa.

Данная библиотека предоставляет 



\section{Выводы и постановка задачи курсового проекта}

Это всегда последний пункт. Здесь, по-первых, приводятся, попунктно, основные вывода из проделанного анализа. Например:

\begin{enumerate}
	\item Выполнен сравнительный анализ таких-то формальных систем с точки зрения применимости к решению такой-то задачи. Ни одна из проанализированных напрямую не подходит, поэтому требуется разработать вариацию на основе системы такой-то.
	\item Были проанализированы варианты программных архитектур на основе систем. С учетом требований к поддержке больших объемов данных и высоких требований к потенциалу модернизируемости, была выбрана за основу такая-то архитектура.
	\item Сравнительный анализ таких-то библиотек показал, что библиотека X проще в использовании, но менее производительна, в то время как библиотека Y обеспечивает высокую производительность, но и требует значительных трудозатрат для использования. В связи с такими-то соображениями были принято решение использовать такую-то библиотеку.
\end{enumerate}

Далее пишется постановка задачи, на основе выданного задания. Это должен быть связный текст в объеме до 1-1,5 страниц. В этом разделе необходимо раскрыть цели и задачи УИРа/диплома.