\chapter{Анализ проблематики задач классификации когнитивных состояний}
\label{chapter1}
\begin{annotation}
	В первой главе подробно рассматриваются теоретические аспекты задачи понижения размерности, задачи классификации (Метод опорных векторов (SVM), нейронные сети, Линейный дискриминантный анализ (ЛДА)) и специфических для проблемной области (фМРТ) подходов к анализу данных. Также описываются программные средства визуализации трёхмерных данных с примерами их использования. (\verb|nilearn.plotting|\cite{10.3389/fninf.2014.00014}, \verb|matplotlib3d|\cite{Hunter:2007}, \verb|NIFTI|, \verb|MITK|\cite{wolf2004medical})
\end{annotation}

\section{Изучение и анализ подходов к классификации когнитивных состояний по данным фМРТ (статическим и динамическим) применительно к задачам медицинской диагностики}
\begin{annotation}
	Для каждого образца объекта или события с известным классом $y$ рассматривается набор наблюдений $x$ (называемых ещё признаками, переменными или измерениями). Набор таких образцов называется обучающей выборкой (или набором обучения, обучением). Задачи классификации состоит в том, чтобы построить хороший прогноз класса $y$ для всякого так же распределённого объекта (не обязательно содержащегося в обучающей выборке), имея только наблюдения $x$.	
\end{annotation}	

В роли объектов выступают пациенты. Признаки характеризуют результаты обследований, симптомы заболевания и применявшиеся методы лечения. Примеры бинарных признаков: пол, наличие головной боли, слабости. Порядковый признак — тяжесть состояния (удовлетворительное, средней тяжести, тяжёлое, крайне тяжёлое). Количественные признаки — возраст, пульс, артериальное давление, содержание гемоглобина в крови, доза препарата. Признаковое описание пациента является, по сути дела, формализованной историей болезни. Накопив достаточное количество прецедентов в электронном виде, можно решать различные задачи:
\begin{itemize}
	\item классифицировать вид заболевания (дифференциальная диагностика);
	\item определять наиболее целесообразный способ лечения;
	\item предсказывать длительность и исход заболевания;
	\item оценивать риск осложнений;
	\item находить синдромы — наиболее характерные для данного заболевания совокупности симптомов.	
\end{itemize}
Ценность такого рода систем в том, что они способны мгновенно анализировать и обобщать огромное количество прецедентов — возможность, недоступная специалисту-врачу.


\section{Сравнительный анализ методов классификации многомерных данных }
\begin{annotation}
	Рассмотрим такие методы как: Метод опорных векторов (SVM), нейронные сети, Линейный дискриминантный анализ (ЛДА)
\end{annotation}

Вначале дадим общее определение \textit{задачи классификации (обучения с учителем)}.

Существует неизвестная целевая зависимость — отображение $y^{*}: X\to Y$, значения которой известны только на объектах конечной обучающей выборки $\left\{(x_i,y_i)| i \in \overline{1,P}\right\}$, где $P$--- количество примеров, $x_i \in  X, y_i \in Y$, $X$-- пространство входных признаков, чаще всего действительное векторное пространство ($\mathbb R^k$), $Y$-- конечное множество классов. Часто множество $Y$ является 2 элементным, в этом случае классификация называется \textit{бинарной}. Требуется построить  алгоритм $\alpha: X\to Y$, который для каждого $x \in \mathcal X$  построить хороший прогноз класса $y$.

Говорят также, что алгоритм должен обладать способностью к обобщению эмпирических фактов, или выводить общее знание (закономерность, зависимость) из частных фактов (наблюдений, прецедентов).

Данная постановка является обобщением классических задач аппроксимации функций. В классической аппроксимации объектами являются действительные числа или векторы. В реальных прикладных задачах входные данные об объектах могуть быть неполными, неточными, неоднородными, нечисловыми. Эти особенности приводят к большому разнообразию методов обучения с учителем.

\subsection{ЛДА}
\textbf{Линейный дискриминантный анализ} (ЛДА), а также связанный с ним \textit{линейный дискриминант Фишера} — методы статистики и машинного обучения, применяемые для нахождения линейных комбинаций признаков, наилучшим образом разделяющих два или более класса объектов или событий. Полученная комбинация может быть использована в качестве линейного классификатора или для сокращения размерности пространства признаков перед последующей классификацией.

Рассмотрим этот метод для случая 2 классов:

При ЛДА предполагается, что функции совместной плотности распределения вероятностей $p(\vec{x}|y=1)$ и $p(\vec{x}|y=0)$ - нормальны. В этих предположениях оптимальное байесовское решение~-- относить точки ко второму классу если отношение правдоподобия ниже некоторого порогового значения $T$: 
$$(\vec{x}-\vec{\mu}_0)^T\Sigma_{y=0}^{-1}(\vec{x}-\vec{\mu}_0)+\ln{|\Sigma _{y=0}|}-(\vec{x}-\vec{\mu}_1)^T\Sigma _{y=1}^{-1}(\vec{x}-\vec{\mu}_1)-\ln{|\Sigma_{y=0}|}<T$$
Если не делается никаких дальнейших предположений, полученную задачу классификации называют квадратичным дискриминантным анализом (\textit{англ. quadratic discriminant analysis, QDA}). В ЛДА делается дополнительное предположение о \textit{гомоскедастичности} (т.е. предполагается, что ковариационные матрицы равны, $\Sigma_{y=0}=\Sigma_{y=1}=\Sigma$) и считается, что ковариационные матрицы имеют полный ранг. При этих предположениях задача упрощается и сводится к сравнению скалярного произведения с пороговым значением 
$$\vec{\omega}\cdot\vec{x}<c $$
\noindent
для некоторой константы $c$, где 
$$\vec{\omega}=\Sigma^{-1}(\vec{\mu_1}-\vec{\mu_0}). $$
\noindent
Это означает, что вероятность принадлежности нового наблюдения x к классу y зависит исключительно от линейной комбинации известных наблюдений.


\subsection{SVM}
Что предпринимать, если данные не гомоскедастичны?
Рассмотрим метод опорных векторов, для чего вначале дадим определение метода.

\textbf{Метод опорных векторов} (\emph {англ. SVM, support vector machine}) — набор схожих алгоритмов обучения с учителем, использующихся для задач классификации и регрессионного анализа. SVM в чистом виде ~-- динейный классификатор.

Основная идея метода — перевод исходных векторов в пространство более высокой размерности и поиск разделяющей гиперплоскости с максимальным зазором в этом пространстве. Две параллельных гиперплоскости строятся по обеим сторонам гиперплоскости, разделяющей классы. Разделяющей гиперплоскостью будет гиперплоскость, максимизирующая расстояние до двух параллельных гиперплоскостей. Алгоритм работает в предположении, что чем больше разница или расстояние между этими параллельными гиперплоскостями, тем меньше будет средняя ошибка классификатора.



\section{Сравнительный анализ программных средств анализа и визуализации трехмерных данных фМРТ и исследование возможности их использования}
\begin{annotation}
	В разделе описаны  различные программные компоненты для визуализации нейро-данных.
\end{annotation}

\subsection{Nilearn}
Данная библиотека предоставляет с лёгкостью использовать продвинутые техники машинного обучения, распознавания образов и статистики на <<нейроданных>> для таких задач как MVPA (многовоксельный анализ закономерностей, \textit{англ. Mutli-Voxel Pattern Analysis}), декодирование, предиктивное моделирование и других.

\texttt{Nilearn} может быть использован для анализа данных фМРТ в состоянии покоя и в случае выполнения испытуемым задач.

\subsection{Analyze}
\texttt{Analyze}~-- ППП, разработанный в \textit{Mayo Clinic} компанией Biomedical Imaging Resource (BIR) для многомерных отображения, обработки и измерения медицинских изображений различного типа. Это коммерческая программа, импользуемая для изучения томорамм, результатов фМРТ, компьютерной томографии, позитрон-эмиссионной томографии (PET).

Автор считает что ПО должно быть свободным и не описывает работу данного пакета.

\subsection{MITK}
Medical Imaging Interaction Toolkit (MITK)-- свободная система с открытым исходным кодом для разработки интерактивного
ПО для обработки медицинских изображений. Внутри себя, MITK содержит Insight Toolkit (ITK), Visualization Toolkit (VTK) и набор инструментов для разработки приложений. Разработана в \textit{German Cancer Research Center
Division of Medical and Biological Informatics}


\section{Выводы и постановка задачи курсового проекта}

Это всегда последний пункт. Здесь, по-первых, приводятся, попунктно, основные вывода из проделанного анализа. Например:

\begin{enumerate}
	\item Выполнен сравнительный анализ таких-то формальных систем с точки зрения применимости к решению такой-то задачи. Ни одна из проанализированных напрямую не подходит, поэтому требуется разработать вариацию на основе системы такой-то.
	\item Были проанализированы варианты программных архитектур на основе систем. С учетом требований к поддержке больших объемов данных и высоких требований к потенциалу модернизируемости, была выбрана за основу такая-то архитектура.
	\item Сравнительный анализ таких-то библиотек показал, что библиотека X проще в использовании, но менее производительна, в то время как библиотека Y обеспечивает высокую производительность, но и требует значительных трудозатрат для использования. В связи с такими-то соображениями были принято решение использовать такую-то библиотеку.
\end{enumerate}

Далее пишется постановка задачи, на основе выданного задания. Это должен быть связный текст в объеме до 1-1,5 страниц. В этом разделе необходимо раскрыть цели и задачи УИРа/диплома.