 \chapter{Алгоритм классификации когнитивных состояний по данным фМРТ на основе метода межиндувидуальных корреляций}


\section{Формальная постановка задачи.}
\begin{annotation}
	Суть алгоритма: посмотреть какие воксели действуют схожим образом для каждого типа стимулов. Для этого применим метод Межиндивидуальных корреляций.
\end{annotation}

\subsection*{Основные Определения и Описание данных}




\section{Алгоритм определения информативных вокселей фМРТ}
\begin{annotation}
В данном разделе описывается алгоритм определения информативных вокселей. Применяются следующие подходы:
\begin{itemize}
	\item Метод межиндивидуальных корреляций(ISC), состоящий в построении корреляционной матрицы для каждого вокселя для всех пар пациентов. Используется коэффициент корреляции Спирмена.
	\item Метод выделения T- статистики
	\item \ldots
\end{itemize}
\end{annotation}

\section{Алгоритм формирования вектора характерных признаков сигналов фМРТ для классификации}
\begin{annotation}
Вектор признаков для кклассификатора формируется на основании результатов предыдущего шага с использованием метода главных компонент. Также убираются из рассмотрения признаки, не меняющиеся в зависимости от класса.
\end{annotation}

\section{Показатели точности классификации}
\begin{annotation}
	В качестве показателей точности используем:
	\begin{itemize}
\item ROC-кривую и её интеграл -- AUC
\item Чувствительность и специфичность:\\ ${\mathit {TPR}}={\mathit {TP}}/P={\mathit {TP}}/({\mathit {TP}}+{\mathit {FN}})$ и \\  $ {\mathit {SPC}}={\mathit {TN}}/N={\mathit {TN}}/({\mathit {TN}}+{\mathit {FP}})$ соответственно

\item \ldots
	\end{itemize}
\end{annotation}

\section{Формальное описание схемы применения алгоритма для классификации когнитивных состояний в режиме реального времени.}

\begin{annotation}
	В этой секции описывается применение алгоритма классификации к данным, поступающим с фМРТ-аппарата.
	Суть раздела: отображение фМРТ--снимку или небольшому набору снимков класса когнитивного состояния. ($\mathrm f (\mathbb N^3\to \mathbb R) \to Y$, где $Y$--конечный набор классов когнитивных состояний). 
\end{annotation}

\section{Выводы}

Необходимо перечислить, какие теоретические результаты были получены с указанием степени новизны. Например: <<Была разработана такая-то модель. Она представляет собой адаптированную версию модели X, в которой уравнение Z заменено на уравнение Z'>>. Еще пример: <<Была предложена такая-то архитектура, она отличается от типовой в том-то и том-то. Это позволяет избежать таких-то проблем.>>. При этом следует заниматься <<высасыванием из пальца>>: <<Поставленная задача является типовой; для ее решения применены стандартные средства (перечислить, какие).>>.