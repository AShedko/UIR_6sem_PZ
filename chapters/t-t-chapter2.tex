 \chapter{Алгоритм классификации когнитивных состояний по данным фМРТ на основе метода межиндувидуальных корреляций}

В этой главе описываются разработанные/модифицированные модели/методы/
алгоритмы, или/и описывается применение известных стандартных методов. Также, 
в конце главы обычно приводится общая архитектура программной системы, 
вытекающая из описанной теории. Приведенные ниже заголовки подразделов так же 
весьма примерные и сильно зависят от особенностей конкретной работы.

Формулы и их части необходимо набирать в математическом режиме
(символ \verb|$|). Во избежание переноса длинных формул между строками их 
стоит размещать по центру колонки, например,

$$S a b c = (\lambda x y z. x z (y z)) a b c = a c (b c),$$

\noindent и, если абзац после формулы продолжается, необходимо использовать 
\verb|\noindent|.

Для набора правил вывода можно использовать пакет \texttt{mathpartir.sty}. 
Правила вывода могут быть вынесены в виде рисунка (см. рис. 
\ref{img:inferrules}).

\begin{figure}[t]
  \centering
    \begin{mathpar}
      \inferrule{
        M \to M'
      }{
        N M \to N M'
      } \quad (\mu) \and 
      \inferrule{
        M \to M'
      }{
        M N \to M' N
      } \quad (\nu) \and
      \inferrule{
        M \to M'
      }{
        \lambda x. M \to \lambda x. M'
      } \quad (\xi)
    \end{mathpar}
  \caption{Правила редукции}
  \label{img:inferrules}
\end{figure}

\section{Формальная постановка задачи.}
\begin{annotation}
	Суть алгоритма: посмотреть какие воксели действуют схожим образом для каждого типа стимулов. Для этого применим метод Межиндивидуальных корреляций.
\end{annotation}

\subsection*{Основные Определения и Описание данных}




\section{Алгоритм определения информативных вокселей фМРТ}

%класт

\section{Алгоритм формирования вектора характерных признаков сигналов фМРТ для классификации}

\section{Показатели точности классификации}

ROC, AUC

\section{Формальное описание схемы применения алгоритма для классификации когнитивных состояний в режиме реального времени.}

\section{Выводы}

Необходимо перечислить, какие теоретические результаты были получены с указанием степени новизны. Например: <<Была разработана такая-то модель. Она представляет собой адаптированную версию модели X, в которой уравнение Z заменено на уравнение Z'>>. Еще пример: <<Была предложена такая-то архитектура, она отличается от типовой в том-то и том-то. Это позволяет избежать таких-то проблем.>>. При этом следует заниматься <<высасыванием из пальца>>: <<Поставленная задача является типовой; для ее решения применены стандартные средства (перечислить, какие).>>.