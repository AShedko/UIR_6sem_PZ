\chapter*{Заключение}
\addcontentsline{toc}{chapter}{Заключение}

В заключении в тезисной форме необходимо отразить результаты работы:

\begin{compactitem}
	\item аналитические (что изучено/проанализировано);
	\item теоретические;
	\item инженерные (что спроектировано);
	\item практические (что реализовано/внедрено).
\end{compactitem}

Примерная формула такая: по каждому указанному пункту приводится по 3-5 результатов, каждый результат излагается в объеме до 5 фраз или предложений.

Также есть смысл привести предполагаемые направления для будущей работы.

\subsection*{Направления будущей работы}

\begin{compactitem}
	\item Изучение применимости нелинейных классификаторов для задач Диагностики.
	\item Использование математических методов (online statistics, Tensor-Train Decomposition\cite{oseledets2011tensor}) для оптимизации времени работы и повышения точности классификации в задачах анализа фМРТ.
\end{compactitem}