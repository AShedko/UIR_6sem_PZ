\chapter*{Реферат}
%\thispagestyle{plain}

Пояснительная записка содержит страниц (из них XX страниц приложений).   Количество использованных источников~-- ХХ. Количество приложений~-- Х.

Ключевые слова: Межиндивидуальная корреляция, Машинное обучение, классификация, фМРТ, Кластеризация.

Целью данной работы является описание применения Межиндивидуальной корреляции для кластеризации признаков при анализе неестественных стимулов в фМРТ.

В первой главе проводится обзор и анализ \dots 

Во второй главе описываются использованные и разработанные/модифицированные методы/модели/алгоритмы \dots. 

В третьей главе приводится описание программной реализации и экспериментальной проверки \dots.

В приложении \ref{code} приведены исходные тексты некоторых программ