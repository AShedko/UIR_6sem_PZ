\setlength\paperheight{297mm}
\setlength\paperwidth{210mm}


\usepackage{polyglossia}
\setmainlanguage{russian}
\setotherlanguages{english}

\usepackage{xunicode} % some extra unicode support
%\usepackage[utf8x]{inputenc}
\usepackage{xltxtra} % \XeLaTeX macro
\usepackage{fontspec}
\defaultfontfeatures{Ligatures=TeX}

%\setromanfont{Charis SIL}
%\setsansfont{Liberation Sans}
%\setmonofont{PT Mono}
%\setmainfont{Liberation Serif} % this allows to use sans-serif as default font

\newfontfamily{\cyrillicfont}{Times New Roman}
\setmainfont[Mapping=tex-text]{Times New Roman}
\newfontfamily{\cyrillicfonttt}{Courier New}
\setmonofont{Courier New}

%нумерация справа и колонтитулы справа вверху
\usepackage{fancyhdr}
\usepackage[left=25mm,right=10mm,top=20mm,bottom=20mm,bindingoffset=0cm]{geometry}%

\usepackage{amsfonts}
\usepackage{amssymb}
\usepackage{amsmath}
\usepackage{amsthm}

\usepackage{calc}
\usepackage{ifthen}
\usepackage{graphicx}
\usepackage{array}
\usepackage{pdfpages}
\usepackage{longtable}
\usepackage{tabu}
\usepackage{indentfirst}
\usepackage[unicode=true]{hyperref}
\usepackage{color}
\usepackage{listingsutf8} % это лучше, чем verbatim
\usepackage{pgf}

\usepackage[singlelinecheck=false,labelsep=endash]{caption}
\captionsetup[table]{justification=justified}
\captionsetup[figure]{justification=centering}

\usepackage{titlesec}
\titleformat{\chapter}[block]{\centering\normalfont\LARGE\bfseries}{\thechapter.}{1ex}{}{}
\titlespacing{\chapter}{0pt}{0em}{2em}

\usepackage[title, titletoc]{appendix}
\addto\captionsrussian{% Replace "english" with the language you use
	\renewcommand{\contentsname}%
	{Содержание}%
}

%\renewcommand{\appendixname}{Приложение}% Change "chapter name" for Appendix chapters
%\renewcommand{\cftchapdotsep}{\cftdotsep}

\usepackage{mathpartir}

\makeatletter
\let\ps@plain\ps@fancy              % Подчиняем первые страницы каждой главы общим правилам
\makeatother
\pagestyle{fancy}
\fancyhf{}
\fancyfoot[C]{\thepage}
\renewcommand{\headrulewidth}{0pt}
\renewcommand{\footrulewidth}{0pt}
\renewcommand{\baselinestretch}{1.5}
\newcommand{\headertext}[1]{\fancyhead[R]{\tiny{#1}}}

%% Список литературы
\makeatletter
\bibliographystyle{utf8gost71s}     % Оформляем список литературы по ГОСТ 7.1
                                    % (ГОСТ Р 7.0.11-2011, 5.6.7)
\renewcommand{\@biblabel}[1]{#1.}   % Заменяем список литературы с квадратных
                                    % скобок на точку
\makeatother

%\frenchspacing %% изменение расстояние до и после точек в ряде случаев

\renewcommand{\theenumi}{\arabic{enumi}}
\renewcommand{\theenumii}{\arabic{enumii}}
\renewcommand{\theenumiii}{\arabic{enumiii}}
\renewcommand{\theenumiv}{\arabic{enumiv}}

\renewcommand{\labelenumi}{\theenumi.}
\renewcommand{\labelenumii}{\theenumi.\theenumii.}
\renewcommand{\labelenumiii}{\theenumi.\theenumii.\theenumiii.}
\renewcommand{\labelenumiv}{\theenumi.\theenumii.\theenumiii.\theenumiv.}


%\newenvironment{annotation}{\textbf{Аннотация.} \textit}{}
\theoremstyle{plain}
\newtheorem*{annotation}{Аннотация}